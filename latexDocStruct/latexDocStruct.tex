%导言区
\documentclass{ctexart}%ctexbook,ctexrep,ctexart

%\usepackage{ctex}

%=============设置标题的格式===============
\ctexset{  %在article形式下无法运行
	section = {
	%format+用于在已有章节的名字格式后面附加内容raggedright:左对齐
		format+ =\zihao{-4} \heiti \raggedright, 
		name = {,、},
		number = \chinese{section},
		%beforeskip 选项用于设置章节标题前的垂直间距 章与章之间的距离
		%afterskip 选项控制章节标题与后面下方之间的距离
		beforeskip = 8.0ex plus 0.2ex minus .2ex,
		afterskip = 1.0ex plus 0.2ex minus .2ex,
		aftername = \hspace{0pt} 
		%aftername将被插入到章节编号与其后的标题内容之间,用于控制格式变
		%换。常用于控制章节编号与标题内容之间的距离,或者控制标题是否另起一行。
	},
	subsection = {
		format+ =\zihao{-5} \heiti \raggedright,
		name = {,、},
		number = \arabic{subsection},
		beforeskip = 1.0ex plus 0.2ex minus .2ex,
		afterskip = 1.0ex plus 0.2ex minus .2ex,
		aftername = \hspace{0pt}
	}
}

%正文区(文稿区)
\begin{document}
	\section{引言}
	生命里,一些缱绻,无论素净,还是喧哗,都已经被岁月赋予了清喜的味道,一些闲词,或清新,或淡雅,总会在某一个回眸的时刻醉了流年,濡湿了柔软的心,冥冥之中,我们沿着呼唤的风声,终于在堆满落花的秋里,再次重逢,念在天涯,心在咫尺,我相信,一米阳光,才是我们最好的距离。
	   
	缘起是诗,缘离是画,这些关于岁月,关于记忆的章节,终会被时光搁置在无法触及的红尘之外,曾经,你我一别经年,可风里,总有一段美丽会与我们不期而遇,一盏琉璃,半杯心悦,端然着那一份醉人的静,这安静行走的流年,总会被过往赋予一份清喜,一份浪漫。
	或许,习惯了着布衣素颜,让清心若雪,不喜张扬,不畏喧哗,守着一怀自己的素韵安静,在自己心中的半亩桃源,修篱种菊,喜欢与山水相依,与流水对话,让文字的墨香,依附在心灵的每一个角落,也喜欢,在闲时,端坐时光一隅,将一本书读到无字,将一盏茶喝到无味,将一个故事看到流泪.……心染尘香,不须有多少的柔情话语去讲,只要能够念起,便是一份温暖。
	
	
	\section{实验方法}
	\section{实验结果}
	\subsection{数据}
	\subsection{图表}
	\subsubsection{实验条件}
	\subsubsection{实验过程}
	\subsection{结果分析}
	\section{结论}
	\section{感谢}
\end{document}

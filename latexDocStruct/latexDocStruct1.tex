%导言区
\documentclass{ctexbook}%ctexbook,ctexrep,ctexart

%\usepackage{ctex}

%=============设置标题的格式===============
\ctexset{  %在article形式下无法运行
	section = {
	%format+用于在已有章节的名字格式后面附加内容raggedright:左对齐
		format+ =\zihao{-4} \heiti \raggedright, 
		name = {,、},
		number = \chinese{section},
		%beforeskip 选项用于设置章节标题前的垂直间距 章与章之间的距离
		%afterskip 选项控制章节标题与后面下方之间的距离
		beforeskip = 8.0ex plus 0.2ex minus .2ex,
		afterskip = 1.0ex plus 0.2ex minus .2ex,
		aftername = \hspace{0pt} 
		%aftername将被插入到章节编号与其后的标题内容之间,用于控制格式变
		%换。常用于控制章节编号与标题内容之间的距离,或者控制标题是否另起一行。
	},
	subsection = {
		format+ =\zihao{-5} \heiti \raggedright,
		name = {,、},
		number = \arabic{subsection},
		beforeskip = 1.0ex plus 0.2ex minus .2ex,
		afterskip = 1.0ex plus 0.2ex minus .2ex,
		aftername = \hspace{0pt}
	}
}

%正文区(文稿区)
\begin{document}
	\tableofcontents %产生目录

	\chapter{绪论}  %用chapter开辟章节
	\section{研究的目的和意义}
	\section{国内外研究现状}
	\subsection{国外研究现状}
	\subsection{国内研究现状}
	\section{研究内容}
	\section{研究方法与技术路线}
	\subsubsection{研究内容}
	\subsection{技术路线}
	
	 \chapter{实验与结果分析}  %用chapter开辟章节
	 \section{引言}
   	 \section{实验方法}
   	 \section{实验结果}
	 \subsection{数据}
	 \subsection{图表}
	 \subsubsection{实验条件}
	 \subsubsection{实验过程}
	 \subsection{结果分析}
	 \section{结论}
	 \section{感谢}
	 	
	 	 
\end{document}
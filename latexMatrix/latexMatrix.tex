%导言区
\documentclass{ctexart}

%\usepackage{ctex}
\usepackage{amsmath}

\newcommand{\adots}{\mathinner{\mkern2mu%
			\raisebox{0.1em}{.}\mkern2mu\raisebox{0.4em}{.}%
			\mkern2mu\raisebox{0.7em}{.}\mkern1mu}}

%正文区
%矩阵环境,用& 分隔列,用\\分隔行
\begin{document}
	\[
	\begin{matrix}
		0 & 1\\
		1 & 0	
	\end{matrix} \qquad
	%pmatricx环境 -矩阵两边加小括号
	\begin{pmatrix}
		0 & -i \\
		i & 0	
	\end{pmatrix}  \qquad
	%bmatricx环境 -矩阵两边加中括号
	\begin{bmatrix}
		0 & -1 \\
		1 & 0
	\end{bmatrix} \qquad
	%Bmatricx环境 -矩阵两边加大括号
	\begin{Bmatrix}
		0 & -1 \\
		1 & 0
	\end{Bmatrix} \qquad
	%vmatricx环境 -矩阵两边加单竖线
	\begin{vmatrix} 
		a & b \\
		c & d
	\end{vmatrix} \qquad
	%Vmatricx环境 -矩阵两边加双竖线
	\begin{Vmatrix} 
		0 & -i \\
		i & 0
	\end{Vmatrix} \qquad
	\]
	
	%可以使用上下标
	\[
	A = \begin{pmatrix}
	a_{11}^2 & a_{12}^2 & a_{13}^2\\
	0 & a_{22} & a_{23}\\
	0 & 0 & a_{33}
	\end{pmatrix}
	\]
	
	%常用省略号:\dots(横向)=cdots  \vdots(竖向) \ddots(右斜向) \adots(自定义左斜向)
	\[
	A = \begin{bmatrix}
	a_{11} & \dots & a_{1n}\\
	\adots & \ddots & \vdots\\
	0 & & a_{nn}
	\end{bmatrix}_{n \times n} %times命令用于排版乘号
	\]
	
	%分块矩阵(矩阵嵌套)
	\[
	\begin{pmatrix}
	\begin{matrix} 	1 & 0\\	0 & 1	\end{matrix}	& \text{\Large 0} \\
	\text{\Large 0} & \begin{matrix} 1 & 0\\ 0 & -1	\end{matrix}
	\end{pmatrix}
	\]
		
		
	%三角矩阵
	\[
	\begin{pmatrix}
	a_{11} & a_{12} & \dots & a_{1n}\\
	& a_{22} & \cdots & a_{2n}\\
	&        & \ddots & \vdots \\
	\multicolumn{2}{c}{\raisebox{1.3ex}[0pt]{\Huge 0}} &   & a_{nn}
	\end{pmatrix}
	\]
	
	
	%跨列的省略号:\hdotsfor{columns}
	\[
	\begin{pmatrix}
	1 & \frac 12 &\dots & \frac 1n\\
	\hdotsfor{4}\\
	m & \frac m2 & \dots & \frac mn
	\end{pmatrix}
	\]
	
	%行内小矩阵(smallmatrix)环境
	复数 $z = (x,y)$也可以用矩阵
	\begin{math}
	\left(% 需要手动加上左括号)
	\begin{smallmatrix}
	x & -y \\ y & x
	\end{smallmatrix}
	\right) %需要手动加上右括号
	\end{math}来表示
	
	
	%array环境(类似于表格环境tabular)
	\[
	\begin{array}{r|r}
	\frac{1}{2} & 0 \\  %\frac{1}{2}=\frac 12
	\hline
	0 & -\frac{a}{bc}
	\end{array}
	\]
	
\end{document}